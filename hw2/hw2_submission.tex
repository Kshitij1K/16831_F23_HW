\documentclass{article}
\usepackage{xcolor}
\usepackage{titleps}
\usepackage[letterpaper, margin=0.95in]{geometry}
\usepackage{url}
\usepackage{amsmath}
\usepackage{amssymb}
\usepackage{wrapfig}
\usepackage{float}
\usepackage{mathtools}
\usepackage{enumitem}
\usepackage{tabu}
\usepackage{parskip}
\usepackage{natbib}
\usepackage{listings}

\usepackage[many]{tcolorbox}
\usepackage{minted}
\setminted[python]{
	% frame=single,
	% linenos,
    xleftmargin=0.475em,
    baselinestretch=1.2,
}
% https://tex.stackexchange.com/a/569249
\setcounter{secnumdepth}{5}
\setcounter{tocdepth}{5}
\makeatletter
\newcommand\subsubsubsection{\@startsection{paragraph}{4}{\z@}{-2.5ex\@plus -1ex \@minus -.25ex}{1.25ex \@plus .25ex}{\normalfont\normalsize\bfseries}}
\newcommand\subsubsubsubsection{\@startsection{subparagraph}{5}{\z@}{-2.5ex\@plus -1ex \@minus -.25ex}{1.25ex \@plus .25ex}{\normalfont\normalsize\bfseries}}
\makeatother

\usepackage{hyperref}
\usepackage[color=red]{todonotes}
\usepackage{forest}
\definecolor{light-yellow}{HTML}{FFE5CC}

\newpagestyle{ruled}
{\sethead{CMU 16-831}{Introduction to Robot Learning }{Fall 2023}\headrule
  \setfoot{}{}{}}
\pagestyle{ruled}

\renewcommand\makeheadrule{\color{black}\rule[-.75\baselineskip]{\linewidth}{0.4pt}}
\renewcommand*\footnoterule{}

\newtcolorbox[]{answer}[1][]{
    % breakable,
    enhanced,
    nobeforeafter,
    colback=white,
    title=Your Answer,
    sidebyside align=top,
    box align=top,
    #1
}



\begin{document}

\lstset{basicstyle = \ttfamily,columns=fullflexible,
backgroundcolor = \color{light-yellow}
}

\begin{centering}
    {\Large Assignment 2: Policy Gradient} \\
    \vspace{.25cm}
    % \textbf{Due September 13, 11:59 pm} \\
\end{centering}
\vspace{0.25cm}

\textbf{Andrew ID:} \texttt{Write your Andrew ID here.} \\
\textbf{Collaborators:} \texttt{Write the Andrew IDs of your collaborators here (if any).}\\ 
\textbf{NOTE:} Please do \textbf{NOT} change the sizes of the answer blocks or plots.

\setcounter{section}{4}
\section{Small-Scale Experiments}

\subsection{Experiment 1 (Cartpole) -- \lbrack25 points total\rbrack}

\subsubsection{Configurations}
\begin{answer}[title=Q5.1.1,height=6cm,width=\linewidth]
% TODO
\begin{minted}
[framesep=2mm, fontsize=\scriptsize, breaklines]
{bash}
python rob831/scripts/run_hw2.py --env_name CartPole-v0 -n 100 -b 1000 \
    -dsa --exp_name q1_sb_no_rtg_dsa

python rob831/scripts/run_hw2.py --env_name CartPole-v0 -n 100 -b 1000 \
    -rtg -dsa --exp_name q1_sb_rtg_dsa

python rob831/scripts/run_hw2.py --env_name CartPole-v0 -n 100 -b 1000 \
    -rtg --exp_name q1_sb_rtg_na

python rob831/scripts/run_hw2.py --env_name CartPole-v0 -n 100 -b 5000 \
    -dsa --exp_name q1_lb_no_rtg_dsa

python rob831/scripts/run_hw2.py --env_name CartPole-v0 -n 100 -b 5000 \
    -rtg -dsa --exp_name q1_lb_rtg_dsa

python rob831/scripts/run_hw2.py --env_name CartPole-v0 -n 100 -b 5000 \
    -rtg --exp_name q1_lb_rtg_na
\end{minted}
\end{answer}

\subsubsection{Plots}

\subsubsubsection{Small batch -- \lbrack5 points\rbrack}
\begin{answer}[title=Q5.1.2.1,height=9.5cm,width=\linewidth]
% TODO
\centering
\includegraphics[height=8cm]{example-image-a}
\end{answer}

\subsubsubsection{Large batch -- \lbrack5 points\rbrack}
\begin{answer}[title=Q5.1.2.2,height=9.5cm,width=\linewidth]
% TODO
\centering
\includegraphics[height=8cm]{example-image-a}
\end{answer}

\subsubsection{Analysis}

\subsubsubsection{Value estimator -- \lbrack5 points\rbrack}
\begin{answer}[title=Q5.1.3.1,height=4cm,width=\linewidth]
% TODO
\end{answer}

\subsubsubsection{Advantage standardization -- \lbrack5 points\rbrack}
\begin{answer}[title=Q5.1.3.2,height=4cm,width=\linewidth]
% TODO
\end{answer}

\subsubsubsection{Batch size -- \lbrack5 points\rbrack}
\begin{answer}[title=Q5.1.3.1,height=4cm,width=\linewidth]
% TODO
\end{answer}

\subsection{Experiment 2 (InvertedPendulum) -- \lbrack15 points total\rbrack}

\subsubsection{Configurations -- \lbrack5 points\rbrack}
\begin{answer}[title=Q5.2.1,height=10cm,width=\linewidth]
% TODO
\begin{minted}
[framesep=2mm, fontsize=\scriptsize, breaklines]
{bash}
python rob831/scripts/run_hw2.py --env_name InvertedPendulum-v4 \
    --ep_len 1000 --discount 0.9 -n 100 -l 2 -s 64 -b <b*> -lr <r*> -rtg \
    --exp_name q2_b<b*>_r<r*>
\end{minted}
\end{answer}

\subsubsection{smallest \textbf{b*} and largest \textbf{r*} (same run) -- \lbrack5 points\rbrack}
\begin{answer}[title=Q5.2.2,height=4cm,width=\linewidth]
% TODO
\end{answer}

\subsubsection{Plot -- \lbrack5 points\rbrack}
\begin{answer}[title=Q5.2.3,height=10cm,width=\linewidth]
% TODO
\centering
\includegraphics[height=8cm]{example-image-a}
\end{answer}

\setcounter{section}{6}
\section{More Complex Experiments}

\subsection{Experiment 3 (LunarLander) -- \lbrack10 points total\rbrack}

\subsubsection{Configurations}
\begin{answer}[title=Q7.1.1,height=6cm,width=\linewidth]
\begin{minted}
[framesep=2mm, fontsize=\scriptsize, breaklines]
{bash}
python rob831/scripts/run_hw2.py \
    --env_name LunarLanderContinuous-v4 --ep_len 1000
    --discount 0.99 -n 100 -l 2 -s 64 -b 40000 -lr 0.005 \
    --reward_to_go --nn_baseline --exp_name q3_b40000_r0.005
\end{minted}
\end{answer}

\subsubsection{Plot -- \lbrack10 points\rbrack}
\begin{answer}[title=Q7.1.2,height=10cm,width=\linewidth]
% TODO
\centering
\includegraphics[height=8cm]{example-image-a}
\end{answer}

\subsection{Experiment 4 (HalfCheetah) -- \lbrack30 points\rbrack}

\subsubsection{Configurations}
\begin{answer}[title=Q7.2.1,height=10cm,width=\linewidth]
\begin{minted}
[framesep=2mm, fontsize=\scriptsize, breaklines, escapeinside=||, mathescape=true]
{python}
# $b \in [10000, 30000, 50000], r\in [0.005, 0.01, 0.02]$
python rob831/scripts/run_hw2.py --env_name HalfCheetah-v4 --ep_len 150 \
    --discount 0.95 -n 100 -l 2 -s 32 -b <b> -lr <r> -rtg --nn_baseline \
    --exp_name q4_search_b<b>_lr<r>_rtg_nnbaseline
\end{minted}
\end{answer}

\subsubsection{Plot -- \lbrack10 points\rbrack}
\begin{answer}[title=Q7.2.2,height=10cm,width=\linewidth]
% TODO
\centering
\includegraphics[height=8cm]{example-image-a}
\end{answer}

\subsubsection{Optimal b* and r* -- \lbrack3 points\rbrack}
\begin{answer}[title=Q7.2.3,height=4cm,width=\linewidth]
% TODO
\end{answer}

\subsubsection{Describe how b* and r* affect task performance -- \lbrack7 points\rbrack}
\begin{answer}[title=Q7.2.4,height=4cm,width=\linewidth]
% TODO
\end{answer}

\subsubsection{Configurations with optimal b* and r* -- \lbrack3 points\rbrack}
\begin{answer}[title=Q7.2.5,height=6cm,width=\linewidth]
% TODO
\begin{minted}
[framesep=2mm, fontsize=\scriptsize, breaklines]
{bash}
python rob831/scripts/run_hw2.py --env_name HalfCheetah-v4 --ep_len 150 \
    --discount 0.95 -n 100 -l 2 -s 32 -b <b*> -lr <r*> \
    --exp_name q4_b<b*>_r<r*>

python rob831/scripts/run_hw2.py --env_name HalfCheetah-v4 --ep_len 150 \
    --discount 0.95 -n 100 -l 2 -s 32 -b <b*> -lr <r*> -rtg \
    --exp_name q4_b<b*>_r<r*>_rtg

python rob831/scripts/run_hw2.py --env_name HalfCheetah-v4 --ep_len 150 \
    --discount 0.95 -n 100 -l 2 -s 32 -b <b*> -lr <r*> --nn_baseline \
    --exp_name q4_b<b*>_r<r*>_nnbaseline

python rob831/scripts/run_hw2.py --env_name HalfCheetah-v4 --ep_len 150 \
    --discount 0.95 -n 100 -l 2 -s 32 -b <b*> -lr <r*> -rtg --nn_baseline \
    --exp_name q4_b<b*>_r<r*>_rtg_nnbaseline
\end{minted}
\end{answer}

\subsubsection{Plot for four runs with optimal b* and r* -- \lbrack7 points\rbrack}
\begin{answer}[title=Q7.2.6,height=10cm,width=\linewidth]
% TODO
\centering
\includegraphics[height=8cm]{example-image-a}
\end{answer}

\section{Implementing Generalized Advantage Estimation}

\subsection{Experiment 5 (Hopper) -- \lbrack20 points\rbrack}

\subsubsection{Configurations}
\begin{answer}[title=Q8.1.1,height=4cm,width=\linewidth]
\begin{minted}
[framesep=2mm, fontsize=\scriptsize, breaklines, escapeinside=||, mathescape=true]
{python}
# $\lambda \in [0,0.95,0.99,1]$
python rob831/scripts/run_hw2.py \
    --env_name Hopper-v4 --ep_len 1000
    --discount 0.99 -n 300 -l 2 -s 32 -b 2000 -lr 0.001 \
    --reward_to_go --nn_baseline --action_noise_std 0.5 --gae_lambda <|$\lambda$|> \
    --exp_name q5_b2000_r0.001_lambda<|$\lambda$|>
\end{minted}
\end{answer}

\subsubsection{Plot -- \lbrack13 points\rbrack}
\begin{answer}[title=Q8.1.2,height=10cm,width=\linewidth]
% TODO
\centering
\includegraphics[height=8cm]{example-image-a}
\end{answer}

\subsubsection{Describe how $\lambda$ affects task performance -- \lbrack7 points\rbrack}
\begin{answer}[title=Q8.1.3,height=4cm,width=\linewidth]
% TODO
\end{answer}

\clearpage

\section{Bonus! (optional)}

\subsection{Parallelization -- \lbrack15 points\rbrack}
\begin{answer}[title=Q9.1,height=4cm,width=\linewidth]
% TODO (optional)
Difference in training time: 
\vspace{1.0cm}
\begin{minted}
[framesep=2mm, fontsize=\scriptsize, breaklines]
{bash}
python rob831/scripts/run_hw2.py \
\end{minted}
\end{answer}

\subsection{Multiple gradient steps -- \lbrack5 points\rbrack}
\begin{answer}[title=Q9.1,height=14cm,width=\linewidth]
% TODO (optional)
\centering
\includegraphics[height=8cm]{example-image-a}

\vspace{1.0cm}
\begin{minted}
[framesep=2mm, fontsize=\scriptsize, breaklines]
{bash}
python rob831/scripts/run_hw2.py \
\end{minted}

\end{answer}

\end{document}

